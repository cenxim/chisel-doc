\documentclass[10pt,landscape]{article}
\usepackage{multicol}
\usepackage[landscape]{geometry}
\usepackage[procnames]{listings}
\usepackage[parfill]{parskip}
\usepackage{fixltx2e}

% "define" Scala
\usepackage[T1]{fontenc}  
\usepackage[scaled=0.82]{beramono}  
\usepackage{microtype} 

\sbox0{\small\ttfamily A}
\edef\mybasewidth{\the\wd0 }

\lstdefinelanguage{scala}{
  morekeywords={abstract,case,catch,class,def,%
    do,else,extends,false,final,finally,%
    for,if,implicit,import,match,mixin,%
    new,null,object,override,package,%
    private,protected,requires,return,sealed,%
    super,this,throw,trait,true,try,%
    type,val,var,while,with,yield},
  sensitive=true,
  morecomment=[l]{//},
  morecomment=[n]{/*}{*/},
  morestring=[b]",
  morestring=[b]',
  morestring=[b]"""
}

\usepackage{color}
\definecolor{dkgreen}{rgb}{0,0.6,0}
\definecolor{gray}{rgb}{0.5,0.5,0.5}
\definecolor{mauve}{rgb}{0.58,0,0.82}

% Default settings for code listings
\lstset{frame=tb,
  language=scala,
  aboveskip=3mm,
  belowskip=3mm,
  showstringspaces=false,
  columns=fixed, % basewidth=\mybasewidth,
  basicstyle={\small\ttfamily},
  numbers=none,
  numberstyle=\footnotesize\color{gray},
  % identifierstyle=\color{red},
  keywordstyle=\color{blue},
  commentstyle=\color{dkgreen},
  stringstyle=\color{mauve},
  frame=single,
  breaklines=true,
  breakatwhitespace=true,
  procnamekeys={def, val, var, class, trait, object, extends},
  procnamestyle=\ttfamily\color{red},
  tabsize=2
}

\lstnewenvironment{scala}
{\lstset{language=scala}}
{}
\lstnewenvironment{cpp}
{\lstset{language=C++}}
{}
\lstnewenvironment{bash}
{\lstset{language=bash}}
{}
\lstnewenvironment{verilog}
{\lstset{language=verilog}}
{}



% Remove section numbering
\setcounter{secnumdepth}{0}

\geometry{top=1cm,left=1cm,right=1cm,bottom=1cm}

\pagestyle{empty}

\begin{document}
\begin{multicols}{3}

\begin{center}
\Large{Chisel Cheat Sheet}
\end{center}

\renewcommand{\tabcolsep}{.5mm}

\section{Notation}
\verb$c, x, y$ are Chisel \verb$Data$s; \verb$n, m$ are Scala \verb$Int$s \newline
\verb$wx, wy$ are the widths of \verb$x, y$ (respectively) \newline
\verb$minVal(x)$, \verb$maxVal(x)$ are the min or max of \verb$x$ \newline
\verb$s$ is a Scala string; \verb$b$ is a Scala boolean \newline
\verb$[ ... ]$ in functions are optional arguments

\section{Basic Operators}
\begin{tabular}{l l}
\verb$val x = UInt()$ & Allocate \verb$a$ as wire of \verb$UInt()$ \\
\verb$x := y$ & Assign \verb$y$ to wire \verb$x$ \\
\verb$x <> y$ & Connect \verb$x$ and \verb$y$, inferring direction \\
\end{tabular}

\section{Basic Data Types}
\subsection{Bool Constructors}
\begin{tabular}{l l l}
\verb$Bool()$ & Create an un-assigned wire \\
\verb$Bool(b)$ & Create a literal \\
\end{tabular}

\subsection{Bits, SInt, UInt Constructors}
\verb$Bits(x:Int|String, width:Int)$ \newline
\verb$UInt(x:Int|String, width:Int)$ \newline
\verb$SInt(x:Int|String, width:Int)$ \newline
\begin{tabular}{l l l}
& \verb$x$ & create a literal from int or parsed string \\
& & or declare unassigned if missing \\
& \verb$width$ & bit width (or inferred) \\
\end{tabular}

\subsection{Bits, UInt, SInt Casts}
\begin{tabular}{l l}
\verb$UInt$ $\rightarrow$ \verb$SInt$ & Zero-extend to SInt \\
(all others) & Reinterpret cast \\
\end{tabular}

\subsection{Bool Operators}
\begin{tabular}{l l l}
Chisel & Explanation & Width \\
\hline
\hline
\verb$!x$ & Logical NOT & \verb$1$ \\
\verb$x && y$ & Logical AND & \verb$1$ \\
\verb$x || y$ & Logical OR & \verb$1$ \\
\end{tabular}

\subsection{Bits Operators}
\begin{tabular}{l l l}
Chisel & Explanation & Width \\
\hline
\hline
\verb$x(n)$ & Extract bit (0 is LSB) & \verb$1$ \\
\verb$x(n, m)$ & Extract bitfield & \verb$n - m + 1$ \\
\verb$x << y$ & Left shift & \verb$wx + maxVal(y)$ \\
\verb$x >> y$ & Right shift & \verb$wx - minVal(y)$ \\
\verb$x << n$ & Left shift & \verb$wx + n$ \\
\verb$x >> n$ & Right shift & \verb$wx - n$ \\
\verb$Fill(n, x)$ & Replicate \verb$x$ \verb$n$ times & \verb$n * wx$ \\
\verb$Cat(x, y)$ & Concatenate bitfields & \verb$wx + wy$ \\
\verb$Mux(c, x, y)$ & If \verb$c$ then \verb$x$ else \verb$y$ & \verb$max(wx, wy)$ \\
\hline
\verb$~x$ & Bitwise NOT & \verb$wx$ \\
\verb$x & y$ & Bitwise AND & \verb$max(wx, wy)$ \\
\verb$x | y$ & Bitwise OR & \verb$max(wx, wy)$ \\
\verb$x ^ y$ & Bitwise XOR & \verb$max(wx, wy)$ \\
\hline
\verb$x === y$ & Equality & \verb$1$ \\
\verb$x != y$ & Inequality & \verb$1$ \\
\hline
\verb$andR(x)$ & AND-reduce & \verb$1$ \\
\verb$orR(x)$ & OR-reduce & \verb$1$ \\
\verb$xorR(x)$ & XOR-reduce & \verb$1$ \\
\end{tabular}

\subsection{UInt, SInt Operators}
Bitwidths only valid for UInt operations

\begin{tabular}{l l l}
Chisel & Explanation & Width \\
\hline
\hline
\verb$x + y$ & Addition & \verb$max(wx, wy)$ \\
\verb$x - y$ & Subtraction & \verb$max(wx, wy)$ \\
\verb$x * y$ & Multiplication & \verb$wx + wy$ \\
\verb$x / y$ & Division & \verb$wx$ \\
\verb$x % y$ & Modulus & \verb$bits(maxVal(y) - 1)$ \\
\hline
\verb$x > y$ & Greater than & \verb$1$ \\
\verb$x >= y$ & Greater than or equal & \verb$1$ \\
\verb$x < y$ & Less than & \verb$1$ \\
\verb$x <= y$ & Less than or equal & \verb$1$ \\
\hline
\verb$x >> y$ & Arithmetic right shift & \verb$wx - minVal(y)$ \\
\verb$x >> n$ & Arithmetic right shift & \verb$wx - n$ \\
\end{tabular}

\section{Helpers}
\subsection{When}
Use \verb$when$ to execute statements conditionally \newline
\verb$when$ behaves similarly to Verilog \newline \verb$always @(posedge clk)$
\begin{scala}
when(condition1) {
  // run if condition1 true
} .elsewhen(condition2) {
  // run if condition2 true
} .unless(condition3) {
  // run if condition3 false
} .otherwise {
  // run if none of the above true
}
\end{scala}

\subsection{Switch}
Use \verb$switch$ to execute statements conditionally \newline
on the value of a wire \newline
\begin{scala}
switch(x) {
  is(value1) {
    // run if x === value1
  } is(value2) {
    // run if x === value2
  }
}
\end{scala}

\subsection{Enum}
Use \verb$enum$s to generate list values \newline
\verb$val s1::s2::$ ... \verb$::sn::Nil$ \newline
\verb$ := Enum(nodeType:UInt, n:Int)$ \newline
\begin{tabular}{l l l}
& \verb$s1$, \verb$s2$, ..., \verb$sn$ & will be created with distinct values \\
& \verb$nodeType$ & type of \verb$s1$, \verb$s2$, ..., \verb$sn$ \\
& \verb$n$ & element count \\
\end{tabular}

\section{Aggregate Types}
\subsection{Bundle}

\subsection{Vec}
\verb$Vec$ contain an indexable vector of \verb$Data$ types
\subsubsection{Constructor}
\verb$val myVec = Vec.fill(n:Int) {gen:Data}$ \newline
\begin{tabular}{l l l}
& \verb$n$ & vector depth (elements) \\
& \verb$gen$ & element data type \\
\end{tabular}
\subsubsection{Reading}
\verb$val readVal = myVec(ind:Data)$ (index by wire) \newline
\verb$val readVal = myVec(idx:Int)$ (static index) \newline
\subsubsection{Writing}
\verb$myVec(ind:Data) := y$ (index by wire) \newline
\verb$myVec(idx:Int) := y$ (static index) \newline

\section{State Elements}
\subsection{Registers}
\subsubsection{Constructor}
\verb$val reg = Reg([outType:Data], [next:Data],$ \newline
\verb$              [init:Data])$ \newline
\begin{tabular}{l l l}
& \verb$outType$ & register type (or inferred) \\
& \verb$next$ & update value every clock \\
& \verb$init$ & initialization value on reset \\
\end{tabular}
\subsubsection{Updating}
\verb$reg := x$ latches \verb$x$ into \verb$reg$ on the next clock \newline
The last update (lexically, per clock) runs

\subsection{Read-Write Memory}
\subsubsection{Constructor}
\verb$val mem = Mem(out:Data, n:Int,$ \newline
\verb$              seqRead:Boolean)$ \newline
\begin{tabular}{l l l}
& \verb$out$ & memory element type \\
& \verb$n$ & memory depth (elements) \\
& \verb$seqRead$ & only update reads on clock edge \\
\end{tabular}
\subsubsection{Reading}
\verb$val readVal = mem(addr:UInt|Int)$ \newline
Synchronous read: assign output to \verb$Reg$
\subsubsection{Writing}
\verb$mem(addr:UInt|Int) := y$ \newline

\subsubsection{}

\section{Modules}

\section{Scala Hardware Generation}
\subsection{For}

\subsection{If}

\subsection{Functional Abstraction}

\section{Standard Library}
\subsection{Decoupled}
\subsection{PipeIO}
\subsection{Arbiter}
\subsection{RRAArbiter}


\end{multicols}
\end{document}
