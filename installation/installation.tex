\documentclass[10pt]{article}
\setlength\textwidth{6.875in}
\setlength\textheight{8.875in}
% set both margins to 2.5 pc
\setlength{\oddsidemargin}{-0.1875in}% 1 - (8.5 - 6.875)/2
\setlength{\evensidemargin}{-0.1875in}
\setlength{\marginparwidth}{0pc}
\setlength{\marginparsep}{0pc}%
\setlength{\topmargin}{0in} \setlength{\headheight}{0pt}
\setlength{\headsep}{0pt}
\setlength{\footskip}{37pt}%
%\setlength{\columnsep}{0.3125in}
%\setlength{\columnwidth}{3.28125in}% (6.875 - 0.3125)/2 = 3.28125in
\setlength{\parindent}{1pc}
\newcommand{\myMargin}{1.00in}
\usepackage[top=\myMargin, left=\myMargin, right=\myMargin, bottom=\myMargin, nohead]{geometry}
\usepackage{epsfig,graphicx}
\usepackage{palatino}
\usepackage{fancybox}

\newenvironment{commentary}
{ \vspace{-0.1in}
  \begin{quotation}
  \noindent
  \small \em
  \rule{\linewidth}{1pt}\\
}
{
  \end{quotation}
}

% \newenvironment{kode}%
% {\footnotesize
%  %\setlength{\parskip}{0pt}
%   %\setlength{\topsep}{0pt}
%   %\setlength{\partopsep}{0pt}
%  \verbatim}
% {\endverbatim 
% %\vspace*{-0.1in}
%  }

% \newenvironment{kode}%
% {\VerbatimEnvironment
% \footnotesize\begin{Sbox}\begin{minipage}{6in}\begin{Verbatim}}%
% {\end{Verbatim}\end{minipage}\end{Sbox}
% \setlength{\fboxsep}{8pt}\fbox{\TheSbox}}

% \newenvironment{kode}
% {\begin{Sbox}
% \footnotesize
% \begin{minipage}{6in}
%   %\setlength{\parskip}{0pt}
%   %\setlength{\topsep}{0pt}
%   %\setlength{\partopsep}{0pt}
%   \verbatim}
% {\endverbatim 
% \end{minipage}
% \end{Sbox} 
% \fbox{\TheSbox}
%  %\vspace*{-0.1in}
%  }

\title{Chisel Installation}
\author{Jonathan Bachrach \\
EECS Department, UC Berkeley\\
{\tt  \{jrb\}@eecs.berkeley.edu}
}
\date{\today}

\newenvironment{example}{\VerbatimEnvironment\begin{footnotesize}\begin{Verbatim}}{\end{Verbatim}\end{footnotesize}}
\newcommand{\kode}[1]{\begin{footnotesize}{\tt #1}\end{footnotesize}}

\def\code#1{{\tt #1}}

\def\note#1{\noindent{\bf [Note: #1]}}
%\def\note#1{}

\begin{document}
\maketitle{}

\section{Introduction}

This document is a installation guide for {\em Chisel} (Constructing
Hardware In a Scala Embedded Language).  Chisel is a hardware
construction language embedded in the high-level programming language
Scala.  

\section{Github}

\begin{itemize}
\item Get an account on \verb|www.github.com|
\item Register your public key on \verb|github.com|
\item Tell me your github account name
\item run \verb|git clone git@github.com:chiselers/chisel.git|
\end{itemize}

\section{Java}

Install a modern Java runtime and compiler.

\section{SBT}

SBT is the Scala Build Tool and is similar to Maven for Java.  Install
and setup SBT using instructions from:

\begin{verbatim}
https://github.com/harrah/xsbt/wiki/Getting-Started-Setup
\end{verbatim}

SBT has a particular directory structure that we adhere to and
somewhat improve.  Assuming that we have a project named {\em transactors},
then the following would be the template:

\begin{verbatim}
chisel/           # install chisel at same level as your project
transactors/
  chisel -> ../chisel
  sbt/
    project/
      build.scala # edit this as shown below
    chisel -> ../chisel/sbt/chisel/
    transactors/
      src/
        main/
          scala -> ../../../../src
  src/ 
    app.scala    # your source files go here
  emulator/      # your C++ target can go here
\end{verbatim}

The following is the \verb+build.scala+ template:

\begin{verbatim}
import sbt._
import Keys._

object BuildSettings {
  val buildOrganization = "edu.berkeley.cs"
  val buildVersion = "1.1"
  val buildScalaVersion = "2.9.2"

  val buildSettings = Defaults.defaultSettings ++ Seq (
    organization := buildOrganization,
    version      := buildVersion,
    scalaVersion := buildScalaVersion
  )
}

object ChiselBuild extends Build {
  import BuildSettings._

  lazy val chisel = 
    Project("chisel", file("chisel"), settings = buildSettings)
  lazy val transactors =
    Project("transactors", file("transactors"), 
            settings = buildSettings) 
      dependsOn(chisel)
}
\end{verbatim}

\end{document}
