\documentclass[twocolumn, 10pt]{article}
\setlength\textwidth{6.875in}
\setlength\textheight{8.875in}
% set both margins to 2.5 pc
\setlength{\oddsidemargin}{-0.1875in}% 1 - (8.5 - 6.875)/2
\setlength{\evensidemargin}{-0.1875in}
\setlength{\marginparwidth}{0pc}
\setlength{\marginparsep}{0pc}%
\setlength{\topmargin}{0in} \setlength{\headheight}{0pt}
\setlength{\headsep}{0pt}
\setlength{\footskip}{37pt}%
%\setlength{\columnsep}{0.3125in}
%\setlength{\columnwidth}{3.28125in}% (6.875 - 0.3125)/2 = 3.28125in
\setlength{\parindent}{1pc}
\newcommand{\myMargin}{1.00in}
\usepackage[top=\myMargin, left=\myMargin, right=\myMargin, bottom=\myMargin, nohead]{geometry}
\usepackage{epsfig,graphicx}
\usepackage{palatino}
\usepackage{fancybox}
\usepackage{hyperref}
\usepackage[procnames]{listings}

% "define" Scala
\usepackage[T1]{fontenc}  
\usepackage[scaled=0.82]{beramono}  
\usepackage{microtype} 

\sbox0{\small\ttfamily A}
\edef\mybasewidth{\the\wd0 }

\lstdefinelanguage{scala}{
  morekeywords={abstract,case,catch,class,def,%
    do,else,extends,false,final,finally,%
    for,if,implicit,import,match,mixin,%
    new,null,object,override,package,%
    private,protected,requires,return,sealed,%
    super,this,throw,trait,true,try,%
    type,val,var,while,with,yield},
  sensitive=true,
  morecomment=[l]{//},
  morecomment=[n]{/*}{*/},
  morestring=[b]",
  morestring=[b]',
  morestring=[b]"""
}

\usepackage{color}
\definecolor{dkgreen}{rgb}{0,0.6,0}
\definecolor{gray}{rgb}{0.5,0.5,0.5}
\definecolor{mauve}{rgb}{0.58,0,0.82}

% Default settings for code listings
\lstset{frame=tb,
  language=scala,
  aboveskip=3mm,
  belowskip=3mm,
  showstringspaces=false,
  columns=fixed, % basewidth=\mybasewidth,
  basicstyle={\small\ttfamily},
  numbers=none,
  numberstyle=\footnotesize\color{gray},
  % identifierstyle=\color{red},
  keywordstyle=\color{blue},
  commentstyle=\color{dkgreen},
  stringstyle=\color{mauve},
  frame=single,
  breaklines=true,
  breakatwhitespace=true,
  procnamekeys={def, val, var, class, trait, object, extends},
  procnamestyle=\ttfamily\color{red},
  tabsize=2
}

\lstnewenvironment{scala}
{\lstset{language=scala}}
{}
\lstnewenvironment{cpp}
{\lstset{language=C++}}
{}
\lstnewenvironment{bash}
{\lstset{language=bash}}
{}
\lstnewenvironment{verilog}
{\lstset{language=verilog}}
{}



\lstset{basicstyle={\footnotesize\ttfamily}}

\newenvironment{commentary}
{ \vspace{-0.1in}
  \begin{quotation}
  \noindent
  \small \em
  \rule{\linewidth}{1pt}\\
}
{
  \end{quotation}
}

% \newenvironment{kode}%
% {\footnotesize
%  %\setlength{\parskip}{0pt}
%   %\setlength{\topsep}{0pt}
%   %\setlength{\partopsep}{0pt}
%  \verbatim}
% {\endverbatim 
% %\vspace*{-0.1in}
%  }

% \newenvironment{kode}%
% {\VerbatimEnvironment
% \footnotesize\begin{Sbox}\begin{minipage}{6in}\begin{Verbatim}}%
% {\end{Verbatim}\end{minipage}\end{Sbox}
% \setlength{\fboxsep}{8pt}\fbox{\TheSbox}}

% \newenvironment{kode}
% {\begin{Sbox}
% \footnotesize
% \begin{minipage}{6in}
%   %\setlength{\parskip}{0pt}
%   %\setlength{\topsep}{0pt}
%   %\setlength{\partopsep}{0pt}
%   \verbatim}
% {\endverbatim 
% \end{minipage}
% \end{Sbox} 
% \fbox{\TheSbox}
%  %\vspace*{-0.1in}
%  }

\title{Chisel Installation}
\author{Jonathan Bachrach \\
EECS Department, UC Berkeley\\
{\tt  \{jrb\}@eecs.berkeley.edu}
}
\date{\today}

\newenvironment{example}{\VerbatimEnvironment\begin{footnotesize}\begin{Verbatim}}{\end{Verbatim}\end{footnotesize}}
\newcommand{\kode}[1]{\begin{footnotesize}{\tt #1}\end{footnotesize}}

\def\code#1{{\tt #1}}

\def\note#1{\noindent{\bf [Note: #1]}}
%\def\note#1{}

\begin{document}
\maketitle{}

\section{Introduction}

This document is a installation guide for {\em Chisel} (Constructing
Hardware In a Scala Embedded Language).  Chisel is a hardware
construction language embedded in the high-level programming language
Scala.  

\section{Setting Up Chisel}

\subsection{Github}

\begin{itemize}
\item Get an account on \verb|www.github.com|
\item Register your public key on \verb|github.com|
\end{itemize}

\subsection{Install}

To install Chisel, download and run the \verb|install| script 
provided on the website. Note: python must be installed for 
this script to work. The script does the following:
\begin{enumerate}
  \item installs \verb|g++| if necessary
  \item installs Java runtime and compiler if necessary
  \item installs \verb+git+ if necessary
  \item prompts for a location to install Chisel (needs an absolute path)\label{step:chiseldir}
  \item clones Chisel into the location provided by Step~\ref{step:chiseldir}
\end{enumerate}

After running the installation script, you need to set the CHISEL
environment variable in your bashrc file to point to where you
installed Chisel in Step~\ref{step:chiseldir}.

\subsection{Verify}
To verify that everything has been configured correctly, download 
run the \verb|verify| script provided on teh website. The
script will run the compiler on simple use case. If everything
goes correctly, you should find the following files in 
\$CHISEL/emulator:
\begin{itemize}
  \item GCD-emulator.cpp
  \item GCD-makefile
  \item GCD.h
  \item GCD.cpp
\end{itemize}

\section{Github}

The following is the Chisel directory structure:

\begin{bash}
chisel/      # install chisel at same level as your project
  tutorial/      
  src/       # chisel source code 
  sbt/
  doc/
    manual/
    tutorial/
    installation/
\end{bash}

\section{Setting Up Chisel}

\begin{bash}
export CHISEL_EMULATOR_INCLUDE = $CHISEL/csrc
\end{bash}

\section{Getting Started with Tutorial}

Test your configuration as follows:

\begin{bash}
cd $CHISEL/tutorial/sbt
sbt
project tutorial
run gcd
\end{bash}

\noindent
Then test that C++ was generated correctly as follows:

\begin{bash}
cd $CHISEL/tutorial/emulator
make -f GCD-Makefile
GCD
\end{bash}

\section{Creating Your Own Projects}

SBT has a particular directory structure that we adhere to and
somewhat improve.  Assuming that we have a project named {\em gpu},
then the following would be the template:

\begin{bash}
chisel/           # install chisel at same level as your project
  tutorial/
  src/
gpu/
  chisel -> ../chisel
  sbt/
    project/
      build.scala # edit this as shown below
    chisel -> ../chisel/sbt/chisel/
    gpu/
      src/
        main/
          scala -> ../../../../src
  src/ 
    gpu.scala # your source files go here
  emulator/   # your C++ target can go here
\end{bash}

Alternatively, you can use \verb+chisel/tutorial+ as a template making
sure to get the symbolic links correct.
The following is the \verb+build.scala+ template:

\begin{scala}
import sbt._
import Keys._

object BuildSettings {
  val buildOrganization = "edu.berkeley.cs"
  val buildVersion = "1.1"
  val buildScalaVersion = "2.9.2"

  val buildSettings = Defaults.defaultSettings ++ Seq (
    organization := buildOrganization,
    version      := buildVersion,
    scalaVersion := buildScalaVersion
  )
}

object ChiselBuild extends Build {
  import BuildSettings._

  lazy val chisel = 
    Project("chisel", file("chisel"), 
            settings = buildSettings)
  lazy val gpu =
    Project("gpu", file("gpu"), 
            settings = buildSettings) 
      dependsOn(chisel)
}
\end{scala}

\end{document}
