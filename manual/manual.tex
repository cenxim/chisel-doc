\documentclass[10pt,twocolumn]{article}
\setlength\textwidth{6.875in}
\setlength\textheight{8.875in}
% set both margins to 2.5 pc
\setlength{\oddsidemargin}{-0.1875in}% 1 - (8.5 - 6.875)/2
\setlength{\evensidemargin}{-0.1875in}
\setlength{\marginparwidth}{0pc}
\setlength{\marginparsep}{0pc}%
\setlength{\topmargin}{0in} \setlength{\headheight}{0pt}
\setlength{\headsep}{0pt}
\setlength{\footskip}{37pt}%
\setlength{\columnsep}{0.3125in}
\setlength{\columnwidth}{3.28125in}% (6.875 - 0.3125)/2 = 3.28125in
\setlength{\parindent}{1pc}
\newcommand{\myMargin}{1.00in}
\usepackage[top=\myMargin, left=\myMargin, right=\myMargin, bottom=\myMargin, nohead]{geometry}
\usepackage{epsfig,graphicx}
\usepackage{palatino}
\usepackage{fancybox}

\newenvironment{commentary}
{ \vspace{-0.1in}
  \begin{quotation}
  \noindent
  \small \em
  \rule{\linewidth}{1pt}\\
}
{
  \end{quotation}
}

\title{Chisel Manual}
\author{Jonathan Bachrach, Krste Asanovi\'{c}, John Wawrzynek \\
EECS Department, UC Berkeley\\
{\tt  \{jrb|krste|johnw\}@eecs.berkeley.edu}
}
\date{\today}

\newenvironment{example}{\VerbatimEnvironment\begin{footnotesize}\begin{Verbatim}}{\end{Verbatim}\end{footnotesize}}
\newcommand{\kode}[1]{\begin{footnotesize}{\tt #1}\end{footnotesize}}

\def\code#1{{\tt #1}}

\def\note#1{\noindent{\bf [Note: #1]}}
%\def\note#1{}

\begin{document}
\maketitle{}

% TODO: default
% TODO: enum yields Bits
% TODO: why hardware construction languages

\section{Introduction}

This document is a manual for {\em Chisel} (Constructing
Hardware In a Scala Embedded Language).  Chisel is a hardware
construction language embedded in the high-level programming language
Scala.  Chisel is really only a set of special class
definitions, predefined objects, and usage conventions within Scala,
so when you write a Chisel program you are actually writing a Scala
program.  However, for the manual we don't presume that you
understand how to program in Scala.  We will point out necessary Scala
features through the Chisel examples we give, and significant hardware
designs can be completed using only the material contained herein.
But as you gain experience and want to make your code simpler or more
reusable, you will find it important to leverage the underlying power
of the Scala language. We recommend you consult one of the excellent
Scala books to become more expert in Scala programming.

\section{Node}

Chisel user code creates a graph of nodes which Chisel then translates
into Verilog or C++ code.  Nodes are \verb+Nameable+ objects:

\begin{example}
trait Nameable {
  var name: String = "";
  var isNamed = false; 
}
\end{example}

% TODO: named should be changed to isNamed

\noindent
of type \verb+Node+:

\begin{example}
class Node extends Nameable {
  def inputs: ArrayBuffer[Node];
  def consumers: ArrayBuffer[Node];
  def infer: Node => Int;
  def getWidth: Int;
  def getNode: Node;
  def toNode: Node;  
  def fromNode(Node): this.type; 
}
\end{example}

% TODO: toNode SHOULD BE toBits
% TODO: fromNode SHOULD BE fromBits

\noindent
where \verb+inputs+ and \verb+consumers+ are the incoming and
outgoing graph edges.

% explain getNode and types interspersed between nodes
% remove fromNode, toNode

\section{Data}

The graph contains port, operator, and type nodes.  Type nodes are
interspersed between port and operator nodes and allow Chisel code to
check and respond to Chisel types.  Type nodes are erased before
emission to C++ and Verilog.  \verb+Data+ is the top most type node
and the following is the built in type node hierarchy:

\begin{example}
Node
  Data
     Bits
        Bool
         Num
           Fix
           UFix
     Bundle
     Vec
\end{example}

\noindent

\verb+Data+ itself is a node:

% TODO: data literals

\begin{example}
class Data extends Node with Cloneable {
  def flatten: Array[(String, IO)];
  def toFix: Fix;
  def toUFix: UFix;
  def toBits: Bits;
  def toBool: Bool;
  def flip: this.type;
  def asOutput: this.type;
  def asInput: this.type;
  def :=[T <: Data](data: T);
}
\end{example}

\noindent
and delegates a number of methods to its single input.  Type nodes are
cloneable:

\begin{example}
trait Cloneable {
  override def clone(): this.type =
    this.getClass.newInstance.asInstanceOf[this.type];
}
\end{example}

\noindent
and users can override this in their own type nodes (e.g., bundles) in
order to reflect construction parameters that are necessary for cloning.

% why not fromBits ?

\subsection{Bits}

The most basic primitive data node is \verb+Bits+ defined as follows:

\begin{example}
object Bits {
  def apply(width: Int = -1): Bits;
  def apply(value: BigInt, width: Int = -1): Bits;
}
class Bits extends Data {
  def unary_-(): Bits;
  def unary_~(): Bits;
  def andR(): Bool;
  def orR():  Bool;
  def ===(b: Bits): Bool;
  def != (b: Bits): Bool;
  def << (b: UFix): Bits;
  def >> (b: UFix): Bits;
  def &  (b: Bits): Bits;
  def |  (b: Bits): Bits;
  def ^  (b: Bits): Bits;
  def ## (b: Bits): Bits;
  def && (b: Bool): Bool;
  def || (b: Bool): Bool;
}
\end{example}

\noindent
with the simple bit operations.  
N.B., that \verb+##+ is binary
concatenation and \verb+===+ is bitwise comparison with \verb+==+.
Operations produce an
actual operator node and a type node combining the input type nodes.
The \verb+getNode+ operator will skip type nodes and return the first
non type node node.

\begin{example}
object Bool{
  def apply():Bool;
  def apply(value:Boolean): Bool;
}
class Bool extends Bits;
\end{example}

\verb+Num+ is a type node which defines arithmetic operations:

\begin{example}
class Num extends Bits {
  def +(b: Num): Num;
  def *(b: Num): Num;
  def -(b: Num): Num;
  def >(b: Num): Bool;
  def <(b: Num): Bool;
  def <=(b: Num): Bool;
  def >=(b: Num): Bool;
}
\end{example}

\noindent
and \verb+Fix+ and \verb+UFix+ are signed and unsigned fixpoint numbers.

\begin{example}
object Fix {
  def apply(width: Int = -1):Fix;
  def apply(value: BigInt, width: Int = -1): Fix;
}
class Fix extends Num; 

object UFix {
  def apply(width: Int = -1):UFix;
  def apply(value: BigInt, width: Int = -1): UFix;
}
class UFix extends Num; 
\end{example}

% literal constructors

\subsection{Bundles and Vec}

\code{Bundle} and \code{Vec} are classes that allow the user to expand
the set of Chisel datatypes with aggregates of other types.

Bundles group together several named fields of potentially different
types into a coherent unit, much like a \code{struct} in C 

\begin{example}
class Bundle extends Data {
  def elements: ArrayBuffer[(String, Data)];
}
\end{example}

\noindent
A user can get the names and Data corresponding to each element in a
Bundle with the \verb+elements+ method and recursively using the
\verb+flatten+ method. 
% TODO: perhaps flatten should be named allElements
Users can define their own by subclassing \verb+Bundle+ as follows:

\begin{example}
class MyFloat extends Bundle {
  val sign =  Bool();
  val exponent = Bits(width = 8);
  val significand =  Bits(width = 23);
}

val x = new MyFloat()
val xs = x.sign
\end{example}

Vecs create an indexable vector of elements: 

\begin{example}
object Vec {
  def apply[T <: Data](n: Int)(gen: => T): Vec[T];
}

class Vec[T <: Data](n: Int, val gen: () => T) 
    extends Data {
  def apply(idx: UFix): T
  def apply(idx: Int): T
}
\end{example}

\noindent
with \verb+n+ elements of type defined with the \verb+gen+ thunk.
Users can then either dynamically or statically accessed the elements.

\subsection{Bit Width Inference}

% TODO: perhaps IO should be renamed Port

Users are required to set bitwidths of ports and registers, but otherwise,
bit widths on wires are automatically inferred unless set manually by the user.
% TODO: how do you set the width explicitly?
The bit-width inference engine starts from the graph's input ports and 
calculates node output bit widths from their respective input bit widths according to the following set of rules:\\

{\footnotesize
\begin{tabular}{ll}
{\bf operation} & {\bf bit width} \\ 
\verb|z = x + y| & \verb|wz = max(wx, wy) + 1| \\
\verb+z = x - y+ & \verb|wz = max(wx, wy) + 1|\\
\verb+z = x & y+ & \verb+wz = max(wx, wy)+ \\
\verb+z = Mux(c, x, y)+ & \verb+wz = max(wx, wy)+ \\
\verb+z = w * y+ & \verb!wz = wx + wy! \\
\verb+z = x << n+ & \verb!wz = wx + maxNum(n)! \\
\verb+z = x >> n+ & \verb+wz = wx - minNum(n)+ \\
\verb+z = Cat(x, y)+ & \verb!wz = wx + wy! \\
\verb+z = Fill(n, x)+ & \verb+wz = wx * maxNum(n)+ \\
% \verb+z = x < y+ & \verb+<= > >= && || != ===+ & \verb+wz = 1+ \\
\end{tabular}
}
\\[1mm]
\noindent  
where for instance $wz$ is the bit width of wire $z$, and the \verb+&+
rule applies to all bitwise logical operations.

The bit-width inference process continues until no bit width changes.
Except for right shifts by known constant amounts, the bit-width
inference rules specify output bit widths that are never smaller than
the input bit widths, and thus, output bit widths either grow or stay
the same.  Furthermore, the width of a register must be specified by
the user either explicitly or from the bitwidth of the reset value.
From these two requirements, we can show that the bit-width inference
process will converge to a fixpoint.

\section{Wire}

We provide declaration of a wire node that can be used immediately, 
but whose input will be set later.

\begin{example}
object Wire {
  def apply[T <: Data]()(gen: =>T): T;
  def apply[T <: Data](default: T): T;
}

class Wire extends Node with Updateable;
\end{example}

%TODO: change proc to Updateable

\noindent
where \verb+Updateable+ collects updates and code generates a mux
input for the wire:

\begin{example}
trait Updateable extends Node {
  def updates: Queue[(Bool, Node)];
  def genMuxes(default: Node);
  def := (x: Node): this.type;
}
\end{example}

% is updateable 

\subsection{Conditional Updates}

\begin{example}
object when {
  def apply(cond: Bool)(block: => Unit): when;
}
class when (prevCond: Bool) {
  def elsewhen (cond: Bool)(block: => Unit): when;
  def otherwise (block: => Unit): Unit;
}
\end{example}

\begin{example}
object unless {
  def apply(c: Bool)(block: => Unit) = 
    when (!c) { block )
}
\end{example}

\begin{example}
object otherwise {
  def apply(block: => Unit) = 
    when (Bool(true)) { block }
}
\end{example}

\begin{example}
object switch {
  def apply(c: Bits)(block: => Unit): Unit;
}
object is {
  def apply(v: Bits)(block: => Unit);
}
\end{example}

\section{Register}

\begin{example}
class Delay extends Node;

object Reg {
  def apply[T <: Data](data: T): T;
  def apply[T <: Data](data: T, resetVal: T): T;
  def apply[T <: Data](resetVal: T = null)(gen: =>T): T;
}
 
class Reg extends Delay with Updateable {
}
\end{example}

\section{Memory}

\begin{example}
object Mem {
  def apply[T <: Data](depth: Int)(gen: => T): T;
}

class Mem[T <: Data](n: Int, gen: () => T) 
    extends Delay with Updateable {
  def apply(idx: UFix): T;
}

class MemRead[T <: Data]
    (val mem: Mem, val idx: UFix) extends Node;

class MemWrite[T <: Data]
    (val mem: Mem, val idx: UFix, val data: T) extends Node;
\end{example}

\section{Components}

Users write their own components by subclassing Component which is
defined as follows:

\begin{example}
abstract class Component {
  val io: Bundle;
  def compileV: Unit;
  def compileC: Unit;
}
\end{example}

\begin{example}
val io = ...;
\end{example}

\subsection{Ports}

\begin{example}
Wire
  IO
\end{example}

\begin{example}
object Input {
  def apply[T <: Data]()(gen: => T): T;
}

object Output {
  def apply[T <: Data]()(gen: => T): T;
}

class IO extends Wire {
}
\end{example}

% is IO a good name?

\begin{example}
def <> (x: Node);
\end{example}

\subsection{Main}

The user calls chiselMain from their main object:

\begin{example}
object chiselMain {
  def apply[T <: Component]
    (args: Array[String], 
     gen: () => T, 
     scanner: T => TestIO = null, 
     printer: T => TestIO = null): T;
}
\end{example}

Command arguments are as follows:
\begin{tabular}{lll}
\verb+--target-dir+ & target pathname prefix \\
\verb+--gen-harness+ & generate harness file for C++ \\
\verb+--v+ & generate verilog \\
\verb+--vcd+ & enable vcd dumping \\
\verb+--debug+ & put all wires in class file \\
\end{tabular}

\section{C++ Emulator}

The C++ emulator is based on a fast multiword library using
C++ templates.  The \verb+dat_t+ represents multiple \verb+val_t+:

\begin{example}
typedef uint64_t val_t;
typedef int64_t sval_t; 
typedef uint32_t half_val_t;
\end{example}

\begin{example}
template <int w>
class dat_t {
 public:
  const static int n_words;
  inline int width ( void );
  inline int n_words_of ( void );
  inline bool to_bool ( void );
  inline val_t lo_word ( void );
  inline unsigned long to_ulong ( void );
  std::string to_str ();
  static dat_t<w> rand();
  dat_t<w> ();
template <int sw> 
  dat_t<w> (const dat_t<sw>& src);
  dat_t<w> (const dat_t<w>& src);
  dat_t<w> (val_t val);
template <int sw> 
  dat_t<w> mask(dat_t<sw> fill, int n);
template <int dw> 
  dat_t<dw> mask(int n);
template <int n> 
  dat_t<n> mask(void);
  dat_t<w> operator + ( dat_t<w> o );
  dat_t<w> operator - ( dat_t<w> o );
  dat_t<w> operator - ( );
  dat_t<w+w> operator * ( dat_t<w> o );
  dat_t<w+w> fix_times_fix( dat_t<w> o );
  dat_t<w+w> ufix_times_fix( dat_t<w> o );
  dat_t<w+w> fix_times_ufix( dat_t<w> o );
  dat_t<1> operator < ( dat_t<w> o );
  dat_t<1> operator > ( dat_t<w> o );
  dat_t<1> operator >= ( dat_t<w> o );
  dat_t<1> operator <= ( dat_t<w> o );
  dat_t<1> gt ( dat_t<w> o );
  dat_t<1> gte ( dat_t<w> o );
  dat_t<1> lt ( dat_t<w> o );
  dat_t<1> lte ( dat_t<w> o );
  dat_t<w> operator ^ ( dat_t<w> o );
  dat_t<w> operator & ( dat_t<w> o );
  dat_t<w> operator | ( dat_t<w> o );
  dat_t<w> operator ~ ( void );
  dat_t<1> operator ! ( void );
  dat_t<1> operator && ( dat_t<1> o );
  dat_t<1> operator || ( dat_t<1> o );
  dat_t<1> operator == ( dat_t<w> o );
  dat_t<1> operator == ( datz_t<w> o );
  dat_t<1> operator != ( dat_t<w> o );
  dat_t<w> operator << ( int amount );
  dat_t<w> operator << ( dat_t<w> o );
  dat_t<w> operator >> ( int amount );
  dat_t<w> operator >> ( dat_t<w> o );
  dat_t<w> rsha ( dat_t<w> o);
  dat_t<w>& operator = ( dat_t<w> o );
  dat_t<w> fill_bit(val_t bit);
  dat_t<w> fill_byte(val_t byte, int nb, int n);
template <int dw, int n> 
  dat_t<dw> fill( void );
template <int dw, int nw> 
  dat_t<dw> fill( dat_t<nw> n );
template <int dw> 
  dat_t<dw> extract();
template <int dw> 
  dat_t<dw> extract(val_t e, val_t s);
template <int dw, int iwe, int iws> 
  dat_t<dw> extract(dat_t<iwe> e, dat_t<iws> s);
template <int sw> 
  dat_t<w> inject(dat_t<sw> src, val_t e, val_t s);
template <int sw, int iwe, int iws> 
  dat_t<w> inject(dat_t<sw> src, dat_t<iwe> e, dat_t<iws> s);
template <int dw> 
  dat_t<dw> log2();
  dat_t<1> bit(val_t b);
  val_t msb();
template <int iw>
  dat_t<1> bit(dat_t<iw> b)
}
\end{example}

\begin{example}
template <int w, int sw> 
  dat_t<w> DAT(dat_t<sw> dat);
template <int w> 
  dat_t<w> LIT(val_t value);
template <int w> dat_t<w> 
  mux ( dat_t<1> t, dat_t<w> c, dat_t<w> a )
\end{example}

The Chisel compiler compiles top level components into a \verb+mod_t+
class that can be created and executed:

\begin{example}
class mod_t {
 public:
  std::vector< mod_t* > children;
  virtual void init (void) { };
  virtual void clock_lo (dat_t<1> reset) { };
  virtual void clock_hi (dat_t<1> reset) { };
  virtual void print (FILE* f) { };
  virtual bool scan (FILE* f) { return true; };
  virtual void dump (FILE* f, int t) { };
};
\end{example}

\begin{example}
environment variables
\end{example}

Either the Chisel compiler can create a harness or the user can write
a harness themselves.  The following is an example of a harness for a
CPU component:

\begin{example}
#include "cpu.h"

int main (int argc, char* argv) {
  cpu_t* cpu = new Cpu();  
  cpu->init();
  for (size_t t = 0; t < n; t++) {
    cpu->scan();
    cpu->clock_lo();
    cpu->clock_hi();
    cpu->print();
  }
}
\end{example}

\section{Verilog}

\section{Standard Library}

% henry

\section{Acknowlegements}

Many people have helped out in the design of Chisel, and we thank them
for their patience, bravery, and belief in a better way.  Many
Berkeley EECS students in the Isis group gave weekly feedback as the
design evolved including but not limited to Yunsup Lee, Andrew
Waterman, Scott Beamer, Chris Celio, etc.  Yunsup Lee gave us feedback
in response to the first RISC-V implementation, called TrainWreck,
translated from Verilog to Chisel.  Andrew Waterman and Yunsup Lee
helped us get our Verilog backend up and running and Chisel TrainWreck
running on an FPGA.  Brian Richards was the first actual Chisel user,
first translating (with Huy Vo) John Hauser's FPU Verilog code to
Chisel, and later implementing generic memory blocks.  Brian gave many
invaluable comments on the design and brought a vast experience in
hardware design and design tools.  Chris Batten shared his fast
multiword C++ template library that inspired our fast emulation
library.  Huy Vo became our undergraduate research assistant and was
the first to actually assist in the Chisel implementation.  We
appreciate all the EECS students who participated in the Chisel
bootcamp and proposed and worked on hardware design projects all of
which pushed the Chisel envelope.  We appreciate the work that James
Martin and Alex Williams did in writing and translating network and
memory controllers and non-blocking caches.  Finally, Chisel's
functional programming and bit-width inference ideas were inspired by
earlier work on a hardware description language called Gel~\cite{gel} designed in
collaboration with Dany Qumsiyeh and Mark Tobenkin.

% \note{Who else?}

\begin{thebibliography}{50}
\bibitem{gel} Bachrach, J., Qumsiyeh, D., Tobenkin, M. \textsl{Hardware Scripting in Gel}.
in Field-Programmable Custom Computing Machines, 2008. FCCM '08. 16th.
\end{thebibliography}

\end{document}
